\documentclass{article}

\usepackage{amsmath}
\usepackage{amssymb}
\usepackage{graphicx}
\usepackage{booktabs}
\usepackage{geometry}
\usepackage[colorlinks,bookmarksopen,bookmarksnumbered,citecolor=blue,urlcolor=blue,linkcolor=blue]{hyperref}
\usepackage{refcount}
\usepackage[sort&compress,comma]{natbib} 

\graphicspath{{pictures/}}

\geometry{margin=0.5in}

\title{RM1 Assignment 1}

\begin{document}

\section*{Variational Inference}

is proportional to a normal density when considered as a function of $\alpha_j$. We can 
identify the parameters of this normal distribution by completing the square of the 
quadratic expression or, more intuitively from a statistical perspective, recognizing the 
expression as equivalent to two pieces of information, one centered at $y_j$ 
with inverse-variance $\sigma_j^{-2}$ and one centered at $E(\mu)$ with inverse-variance 
$E(\frac{1}{\tau^2})$. 

We combine these by weighting the means and adding the inverse-variances, thus 
getting the following form for the variational Bayes component for $\alpha_j$:
\[
g(\alpha_j) = \mathcal{N}\left(\alpha_j \, \Bigg| \, 
\frac{\frac{1}{\sigma_j^2}y_j + E\left(\frac{1}{\tau^2}\right)E(\mu)}
{\frac{1}{\sigma_j^2} + E\left(\frac{1}{\tau^2}\right)}, \,
\frac{1}{\frac{1}{\sigma_j^2} + E\left(\frac{1}{\tau^2}\right)} 
\right). \tag{13.18}
\]

For $\mu$, we inspect (13.16). Averaging over all the parameters other than $\mu$, the 
expression $E \log p(\theta \mid y)$ has the form 
\[
-\frac{1}{2}E\left(\frac{1}{\tau^2}\right) \sum_{j=1}^8 (E(\alpha_j) - \mu)^2 + \text{const}.
\]
As above, this is the logarithm of a normal density function; the parameters of this distribution 
can be determined by considering it as a combination of 8 pieces of information:
\[
g(\mu) = \mathcal{N}\left(\mu \, \Bigg| \, 
\frac{1}{8} \sum_{j=1}^8 E(\alpha_j), \,
\frac{1}{8 E\left(\frac{1}{\tau^2}\right)} 
\right). \tag{13.19}
\]

Finally, averaging over all parameters other than $\tau$ gives a density function that 
can be recognized as inverse-gamma or, in the parameterization we prefer,
\[
g(\tau^2) = \text{Inv-}\chi^2\left(\tau^2 \, \Bigg| \, 
7, \, 
\frac{1}{7} \sum_{j=1}^8 E\left((\alpha_j - \mu)^2\right) 
\right), \tag{13.20}
\]
with the expectation $E\left((\alpha_j - \mu)^2\right)$ over the approximating distribution $g$.

The above expressions are essentially identical to the derivations of the conditional 
distributions for the Gibbs sampler for the hierarchical normal model in Section 11.6 
and the EM algorithm in Section 13.6, with the only difference being that in the 
8-schools example we assume the data variances $\sigma_j$ are known.

\subsection*{Determining the conditional expectations}
Rewriting the above factors in generic notation, we have:
\[
g(\alpha_j) = \mathcal{N}(\alpha_j \mid M_{\alpha_j}, S^2_{\alpha_j}), \quad j = 1, \ldots, 8, \tag{13.21}
\]
\[
g(\mu) = \mathcal{N}(\mu \mid M_\mu, S^2_\mu), \tag{13.22}
\]
\[
g(\tau^2) = \text{Inv-}\chi^2(\tau^2 \mid T, M_\tau^2). \tag{13.23}
\]

We will need these to get the conditional expectations for each of the above three steps:
\begin{itemize}
    \item To specify the distribution for $\alpha_j$ in (13.18), we need $E(\mu)$, which is $M_\mu$ 
    from (13.22), and $E\left(\frac{1}{\tau^2}\right)$, which is $\frac{1}{M_\tau^2}$ from (13.23).
    \item To specify the distribution for $\mu$ in (13.19), we need $E(\alpha_j)$, which is $M_{\alpha_j}$ 
    from (13.21), and $E\left(\frac{1}{\tau^2}\right)$, which is $\frac{1}{M_\tau^2}$ from (13.23).
    \item To specify the distribution for $\tau$ in (13.20), we need $E\left((\alpha_j - \mu)^2\right)$, 
    which is $(M_{\alpha_j} - M_\mu)^2 + S_{\alpha_j}^2 + S_\mu^2$ from (13.21) and (13.22), and using the 
    assumption that the densities $g$ are independent in the variational approximation.
\end{itemize}



\end{document}
